\documentclass[article, 10pt]{article} 
\usepackage{graphicx, booktabs, multirow, multicol} % Required for inserting images
% \usepackage{biblatex}
% \usepackage{hyperref}
\usepackage[pdftex,bookmarksnumbered,hidelinks,breaklinks]{hyperref}
% \usepackage[hyphens]{url}  %% be sure to specify the option 'hyphens'
\usepackage{pythonhighlight}
\usepackage{caption}
\captionsetup[table]{
  labelsep=newline,
  justification=justified,
  singlelinecheck=false,
  textfont=it,
}
\graphicspath{ {./images/} }
\usepackage{listings}
  \lstdefinestyle{tree}{
    literate=
    {├}{{\smash{\raisebox{-1ex}{\rule{1pt}{\baselineskip}}}\raisebox{0.5ex}{\rule{1ex}{1pt}}}}1 
    {│}{{\smash{\raisebox{-1ex}{\rule{1pt}{\baselineskip}}}\raisebox{0.5ex}{\rule{1ex}{0pt}}}}1
    {─}{{\raisebox{0.5ex}{\rule{1.5ex}{1pt}}}}1 
    {└}{{\smash{\raisebox{0.5ex}{\rule{1pt}{\dimexpr\baselineskip-1.5ex}}}\raisebox{0.5ex}{\rule{1ex}{1pt}}}}1 
  }
\title{FindFirst Image Similarity Search \\ \large Dr. Jian Wu  Information Retrieval}

\author{Raphael J. Sandor }
\date{November 2023}

\begin{document}

\maketitle

\section{Introduction}
The objective of FindFirst Image similarity search is create a search engine that is 
capable of text to image and image to image search while using semantic understanding to 
render the closest image to the provided query. 
The search engine must be able to classify a given image between the nineteen different 
types of figures. With a responsive interface to a users requests, i.e. images return to
the user in less than a second. Other features that were employed in the application was
probabilistic approach to labeling the images to figure classifications as is discussed 
later in this report, as well investigation into improving the classification model. 

% TODO ON SATURDAY 11-18-2023

\section{Data}
% TODO ON SUNDAY 11-19-2023
% DISCUSS ACL FIGURE DATA SET 
% - WHERE IT COMES FROM 
% - THE AMOUNT OF DATA
% - PLEASE REFERENCE THE PAPER 
% - CHALLENGES OF THE DATA
\subsection{About the data}
The ACL dataset consists of two types of data, SciFig and SciFig-pilot which is labeled data. 
SciFig dataset consists of 263,952 unlabelled images spread across nineteen different
figure types which are enumerated in Table \ref{table:label-dist} with their respective distributions: 

\begin{table}
    \begin{tabular}{c|c||c|c}
    \hline
     Class                  &  \%   & Class         &  \% \\ \hline
     Trees                  & 13    & Graphs        & 6 \\  \hline
     Natural Images         & 8     & Tables        & 6 \\  \hline
     Confusion Matrix       & 7     & Screenshots   & 6 \\  \hline
     Pie Charts             & 6     & Bar Charts    & 6 \\  \hline
     NLP text/grammar       & 6     & Box plots     & 2 \\  \hline
     Architecture Diagram   & 6     & Venn Diagram  & 1 \\  \hline
     Algorithm              & 6     & Word Cloud    & 1 \\  \hline
     Neural Networks        & 6     & Pareto        & 1 \\  \hline
     Line Graph             & 6     &               & \\  \hline
    \end{tabular}
    \caption{Distribution \cite{ACL}}
    \label{table:label-dist}
\end{table}

The figures are located under the data/SciFig/png, as shown in Figure \ref{fig:directory-structure}, the metadata is located under data/SciFig/metadata. 

\begin{figure}
    \centering
    \caption{Directory Structure}
    \label{fig:directory-structure}
\begin{lstlisting}[style=tree]
data/
├── SciFig
│   ├── metadata
│   └── png
└── SciFig-pilot
    ├── algorithms
    ├── architecture diagram
    ├── bar charts
    ├── boxplots
    ├── confusion matrix
    ├── graph
    ├── Line graph_chart
    ├── maps
    ├── metadata
    ├── natural images
    ├── neural networks
    ├── NLP text_grammar_eg
    ├── pareto
    ├── pie chart
    ├── png
    ├── scatter plot
    ├── Screenshots
    ├── tables
    ├── trees
    ├── venn diagram
    └── word cloud
\end{lstlisting}
\end{figure}

\subsection{The metadata}
The metadata provided for SciFig dataset included details about location of the detected 
figures in "raw\_detected\_boxes", as well the output of figures which could be either "raw\_pdffigures\_output" or under Figures. Each figure type containing figure type which is either Table or Figure. Name, caption boundary, image text or caption text, page location, uri, page, and dpi. The metadata was extracted in the orginal work using PDFFigures2 and DeepFigures \cite{ACL}.

The fields referenced in FindFirst were name, figure type. The figure type being used to determine if a prediction was needed to be made as tables were always accurately classified. 

\subsection{The pilot-data}
The pilot data consists of 19 different labels differentiated by the directory in which the files images that in each belong. The tree was flattened from it original orientation in which each of the image classes were located under png. The choice to preserve all of the classes and place every classes images also in the png directory was so that the SciFig and SciFig-pilot's directories were the same orientation in the png directory which is useful for testing; requiring only a single path to be changed. 

\section{Project Architecture}
% TUESDAY 11-21

\subsection{Proposed solution}
% WEDNESDAY 11-22

\subsection{Changes in design}
% THURSDAY (THANKSGIVING) 11-23-2023
Create microservice for Pytorch.

\section{Deployment}
% 11-24

\subsection{System Requirements}
% 11-25

\subsection{Performance}
% 11-26

\section{CLIP}
As mentioned in https://github.com/openai/CLIP the original zero-shot clip model has been
trained on 1.28M labelled examples which allows for the user of the model to be able to do
semantic search \cite{OpenAI}. This allows the user of the model to provide an a sentence in 
the from of string characters to the model and receive and vector representing the string. The 
same is true for providing an image, such png or jpeg, to the model. 

This allows the user to use the vector data to compare texts to and image, and use the image in search engines such as elastic search which can use  k-nearest-neighbors (KNN) search. 

\subsection{Sentence Transformers}
The model card that is available on GitHub from OpenAI/CLIP was initially used for the classification of the models
\begin{python}
import torch
import clip
from PIL import Image

device = "cuda" if torch.cuda.is_available() else "cpu"
model, preprocess = clip.load("ViT-B/32", device=device)

image = preprocess(Image.open("CLIP.png")).unsqueeze(0).to(device)
text = clip.tokenize(["a diagram", "a dog", "a cat"]).to(device)

with torch.no_grad():
    image_features = model.encode_image(image)
    text_features = model.encode_text(text)
    
    logits_per_image, logits_per_text = model(image, text)
    probs = logits_per_image.softmax(dim=-1).cpu().numpy()

print("Label probs:", probs)  # prints: [[0.9927937  0.00421068 0.00299572]]
\end{python} \cite{OpenAI}
The model out of the box is as shown above is not optimized for hardware. For example running the model on image to predictions produced 4.25 classifications per second, on a dataset of 264k images this would take nearly seventeen and half hours to complete. 

Pre-optimized solutions were researched to reduce the amount of time to produce classifications on the full dataset. This resulted in testing the use of SBERT CLIP model with their sentence-transformer \cite{SBERT}.
The predictions on the Sentence-Transformer verses the native Pytorch model were within
tolerance of approximately .020-.026 with Sentence-Transformers producing 17.5/its, thereby reducing the prediction time to 4.19 hours.

\subsubsection{Tokenization}
% Discuss Sensitivity to class string.

\subsection{Fine Tuning CLIP}
There is one problem with this, while model itself provides adequate results for general text to image, and image to image translation e.g. "cats on a bed", it doesn't perform well for tasks specialized classification on labels.
Using the zero shot model in the application of classifying images on 19 different labels from ACL Academic Figure set revealed that there is glaring bias to certain classifications. For example, when the figure has more text, such as captions included, the model assumes that the figure contains data about Natural Language Processing (N.L.P.), other issues include types of charts that occupy
a low percentage of the overall dataset such Pareto Charts are occurring frequently in queries. 

The solution is thus to fine tune the model to the dataset of ACLs figures type, the
challenge being that there is a limited number of labelled data. In total there is only
1671 labeled images in the pilot (training) dataset. 

\subsubsection{Creating the labeled data}
As mentioned in the data section of the report the labeled data in under pilot data. To use pilot data it was necessary to clean up the data structure to begin creating labeled data. For example each of the directories represented one of the labels, however the directory names were not a one-to-one with labels that were used through out the rest of the application. For example, "box plot" is a label for box plot figures, the directory in which this labeled data is stored is "boxplot". "NLP text\_grammar\_eg" also needed to be changed to the standard used "natural language processing", and "Screenshots" to "screenshot".

Next a Python Script was created to collect all of the data in each of these directories ignoring the directories that were added during flattening, i.e. png and metadata directories. 



\subsubsection{Loading CLIP and Processor}
The SentenceTransformer wrapper for the model was very efficient but the model in the
SentenceTransformer is not able to directly apply transfer learning. Resulting in the 
application of importing and using torch and clip directly. 

\subsection{Pytorch CLIP Model vs Using Sentence Transformers}

% 11-18


\subsection{Pytorch} 
% 11-23
Discuss using Sentence Transformers. 


\section{Search}
% 11-28
\subsection{Searching}
% 11-29
discus applying filters: https://www.elastic.co/guide/en/elasticsearch/reference/current/filter-search-results.html
  - performance on a post filter vs filter.
\subsubsection{Nested sorting}
% 11-30
\subsection{Scoring}
% 12-01

\subsection{Evaluation}
% 12-02


% This section may end up being covered a earlier
% \section{Model improvements}
% \subsection{Methods}
% \subsection{Evaluation}

\section{Search improvements}
% 12-03
\subsection{Multi modal}
One key feature of FindFirst is that allows for the user to query based not only on the 
query that is semantically 
% 12-04
Showing how the search improves with multi modal 
\subsection{Re-render on users clicking check boxes}
% 12-05
\subsection{Images are cached to be reused in queries}
% 12-05

\subsection{Find Similar}
% 12-05

\section{Lessons learned} 
% 12-05
Using a the image CLIP model to do image classification has limitations. 


\begin{thebibliography}{99}
\bibitem{OpenAI} OpenAI (2023, July 8). CLIP. GitHub; OpenAI. \url{https://github.com/openai/CLIP}
\bibitem{ACL} SciFig: A Scientific Figure Dataset for Figure Understanding. (2022). \url{https://openreview.net/pdf?id=tYxt7Y0os6I}

\bibitem{SBERT} SentenceTransformers Documentation — Sentence-Transformers documentation. (n.d.). www.sbert.net. \url{https://www.sbert.net/}


\end{thebibliography}

\end{document}